Basic Arithmetic Expression Evaluator based on Stacks

\subsection*{Descrição}

O B\+A\+R\+ES (Basic A\+Rithmetic Expression Evaluator based on Stacks) e um avaliador de expressoes baseado em pilhas que recebe um conjunto de expressoes e verifica se as mesmas sao validas retornando o resultado da operacao, caso seja invalida o erro sera mostrado na coluna da espressao. Vale salientar que cada expressao deve estar em uma linha do arquivo para ser valida, quebras de linha na expressao torna-\/a invalida para calculos. Cada valor de expressao tem o tamanho de um inteiro, ou seja, deve estar entre -\/32767 e 32767. Abaixo segue uma tabela de operacoes suportadas pelo B\+A\+R\+ES.

\subsubsection*{Operadores Suportados}

\tabulinesep=1mm
\begin{longtabu} spread 0pt [c]{*3{|X[-1]}|}
\hline
\rowcolor{\tableheadbgcolor}\PBS\centering {\bf Simbolo }&{\bf Operacoes }&\PBS\centering {\bf Precedencia  }\\\cline{1-3}
\endfirsthead
\hline
\endfoot
\hline
\rowcolor{\tableheadbgcolor}\PBS\centering {\bf Simbolo }&{\bf Operacoes }&\PBS\centering {\bf Precedencia  }\\\cline{1-3}
\endhead
\PBS\centering \+\_\+\+\_\+ &Menos unario &\PBS\centering 1 \\\cline{1-3}
\PBS\centering \+\_\+\+\_\+$^\wedge$\+\_\+\+\_\+ &Exponenciacao &\PBS\centering 2 \\\cline{1-3}
\PBS\centering \+\_\+\+\_\+/\+\_\+\+\_\+ &Divisao &\PBS\centering 3 \\\cline{1-3}
\PBS\centering \+\_\+\+\_\+\+\_\+\+\_\+ &Modulo &\PBS\centering 3 \\\cline{1-3}
\PBS\centering \+\_\+\+\_\+$\ast$\+\_\+\+\_\+ &Multiplicacao &\PBS\centering 3 \\\cline{1-3}
\PBS\centering \+\_\+\+\_\+-\/\+\_\+\+\_\+ &Subtracao &\PBS\centering 4 \\\cline{1-3}
\PBS\centering \+\_\+\+\_\++\+\_\+\+\_\+ &Adicao &\PBS\centering 4 \\\cline{1-3}
\PBS\centering \+\_\+\+\_\+()\+\_\+\+\_\+ &Parenteses &\PBS\centering 5 \\\cline{1-3}
\end{longtabu}


\subsection*{Compilação}

g++ -\/std=c++11 -\/pedantic -\/I include/ \hyperlink{drive_8cpp}{src/drive.\+cpp} -\/o bin/exe

\subsection*{Execução}

Dentro da pasta B\+A\+R\+ES execute\+:

\$ ./bin/exe data/data.\+txt

\subsection*{Lista de erros que o programa trata}


\begin{DoxyEnumerate}
\item Numeric constant out of range\+: Um dos operandos da expressao esta fora da faixa permitida. Ex.\+: 1000000 − 2, coluna 1.
\item Ill-\/formed expression or missing term detected\+: Em alguma parte da expressao esta faltando um operando ou existe algum operando em formato errado. Ex.\+: 2+, coluna 3; ou 3 ∗ d, coluna 5.
\item Invalid operand\+: Existe um sımbolo correspondente a um operador que nao esta na lista de operadores validos. Ex.\+: 2 = 3, coluna 3; ou 2.\+3+4, coluna 2.
\item Extraneous symbol\+: Aparentemente o programa encontrou um sımbolo extra “perdido” na expressao. Ex.\+: 2 ∗ 3 4, coluna 7 ou (−3∗4)(10∗5), coluna 7.
\item Mismatch ’)’\+: Existem um parˆentese fechando sem ter um parentese abrindo correspondente. Ex.\+: )2−4, coluna 1; ou 2 − 4), coluna 6; ou 2) − 4. coluna 2.
\item Lost operator\+: Apareceu um operador sem seus operandos. 2 ∗∗ 3,coluna4;ou/5 ∗ 10,coluna 1.
\item Missing closing ‘)’ to match opening ‘)’ at\+: Esta faltando um parentese de fechamento ’)’ para um parentese de abertura ‘(’ correspondente. Ex.\+: ((2\%3) ∗ 8, coluna 1.
\item Division by zero!\+: Houve divisao cujo quociente e zero. Ex.\+: 3/(1 − 1); ou 10/(3 ∗ 3ˆ−2). Nestes casos nao e preciso informar a coluna.
\item Numeric overflow error!\+: Acontece quando uma operacao dentro da expressao ou a expressao inteira estoura o limite das constantes numericas definidos na Secao 1. Ex.\+: 20 ∗ 20000. Nestes casos nao e preciso informar a coluna.
\end{DoxyEnumerate}

\subsubsection*{Limitacoes}


\begin{DoxyItemize}
\item O programa não está tratando unário
\end{DoxyItemize}

\subsection*{Autores\+:}


\begin{DoxyItemize}
\item Adelino Afonso Fernandes Avelino -\/ \href{mailto:adelino-afonso@hotmail.com}{\tt adelino-\/afonso@hotmail.\+com}
\item Irene Ginani Costa Pinheiro -\/ \href{mailto:ireneginani@gmail.com}{\tt ireneginani@gmail.\+com} 
\end{DoxyItemize}